\documentclass{article}
 
\usepackage{mathtools}
\usepackage[utf8]{inputenc}
\usepackage[portuguese]{babel}

\begin{document}
\section{Aparente contradição encontrada com a distribuição de probabilidade da variável aleatória que é o produto de outras duas}


Sejam $X$ e $Y$ duas variáveis contínuas e independentes descritas
pelas funções densidade de probabilidade $f_X$ e $f_Y$.
Segue que a função densidade de probabilidade de $Z=XY$ é:

\begin{equation}
f_Z(z) = \int_{-\infty}^{\infty} f_X(x) f_Y(z/x)\frac{1}{|x|}dx
\end{equation}

\noindent Suponha 
$f_X(x)=\frac{C_1}{x^{\alpha_1}}$
e
$f_Y(y)=\frac{C_2}{y^{\alpha_2}}$
com $C_1, C_2, \alpha_1, \alpha_2$ constantes e não nulos.
Considere também $1 \leq x,y \leq L$.
Aplicando a fórmula inicial:
% \begin{equation}
% \begin{split}
\begin{align}
f_Z(z) = & \int_{-\infty}^{\infty} f_X(x) f_Y(z/x)\frac{1}{|x|}dx & = & \int_{-\infty}^{\infty} f_Y(y) f_X(z/y)\frac{1}{|y|}dy\\
& \int_{-\infty}^{\infty} \frac{C_1}{x^{\alpha_1}} \frac{C_2}{(z/x)^{\alpha_2}}\frac{1}{x}dx & = & \int_{-\infty}^{\infty} \frac{C_2}{y^{\alpha_2}} \frac{C_1}{(z/y)^{\alpha_1}}\frac{1}{y}dy\\
& C_1 C_2 z^{-\alpha_2} \int_{1}^{L} x^{\alpha_2-\alpha_1-1} dx & = & C_1 C_2 z^{-\alpha_1} \int_{1}^{L} y^{\alpha_1-\alpha_2-1} dy\\
& C_1 C_2 z^{-\alpha_2} \left. \frac{x^{\alpha_2-\alpha_1}}{\alpha_2-\alpha_1}\right|_{x=1}^{L} & = & C_1 C_2 z^{-\alpha_1} \left. \frac{y^{\alpha_1-\alpha_2}}{\alpha_1-\alpha_2}\right|_{y=1}^{L}  \\
& z^{-\alpha_2} \left[ \frac{C_1 C_2}{\alpha_2-\alpha_1} (L^{\alpha_2-\alpha_1} -1)\right] & = & z^{-\alpha_1} \left[\frac{C_1 C_2}{\alpha_1-\alpha_2} (L^{\alpha_1-\alpha_2} -1)\right]
\end{align}
\noindent O que é impossível se $\alpha_1 \neq \alpha_2$.

(Não pode haver equivalência entre duas leis de potência de expoentes diferentes,
o que é evidente já na terceira linha com a integral definida.
O único termo não constante na última linha é $z$.)

% \end{split}
% \end{equation}
\end{document}
