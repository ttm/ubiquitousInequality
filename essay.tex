%%%%%%%%%%%%%%%%%%%%%%%%%%%%%%%%%%%%%%%%%
% Thin Sectioned Essay
% LaTeX Template
% Version 1.0 (3/8/13)
%
% This template has been downloaded from:
% http://www.LaTeXTemplates.com
%
% Original Author:
% Nicolas Diaz (nsdiaz@uc.cl) with extensive modifications by:
% Vel (vel@latextemplates.com)
%
% License:
% CC BY-NC-SA 3.0 (http://creativecommons.org/licenses/by-nc-sa/3.0/)
%
%%%%%%%%%%%%%%%%%%%%%%%%%%%%%%%%%%%%%%%%%

%----------------------------------------------------------------------------------------
%   PACKAGES AND OTHER DOCUMENT CONFIGURATIONS
%----------------------------------------------------------------------------------------

\documentclass[a4paper, 11pt]{article} % Font size (can be 10pt, 11pt or 12pt) and paper size (remove a4paper for US letter paper)
\usepackage{hyperref}
\hypersetup{
        colorlinks,
            linkcolor={red!50!black},
                citecolor={blue!50!black},
                    urlcolor={blue!80!black}
                }


\usepackage[portuguese,english]{babel}
\usepackage[utf8]{inputenc}
%\usepackage{float}
\DeclareUnicodeCharacter{202D}{}
\DeclareUnicodeCharacter{202C}{}

%\usepackage{color} % for the notes
\usepackage{xcolor}
\usepackage[protrusion=true,expansion=true]{microtype} % Better typography
%\usepackage{graphicx} % Required for including pictures
%\usepackage{wrapfig} % Allows in-line images
%\usepackage{tocloft}
%\usepackage{multirow}

\usepackage{mathpazo} % Use the Palatino font
%\usepackage[T1]{fontenc} % Required for accented characters
%\linespread{1.05} % Change line spacing here, Palatino benefits from a slight increase by default
%\usepackage{etoolbox}

\newtheorem{theorem}{Theorem}[section]
\newtheorem{lemma}[theorem]{Lemma}
\newtheorem{axiom}[theorem]{Axiom}
\newtheorem{proposition}[theorem]{Proposition}
\newtheorem{corollary}[theorem]{Corollary}

\newenvironment{proof}[1][Proof]{\begin{trivlist}
\item[\hskip \labelsep {\bfseries #1}]}{\end{trivlist}}
	\newenvironment{definition}[1][Definition]{\begin{trivlist}
\item[\hskip \labelsep {\bfseries #1}]}{\end{trivlist}}
	\newenvironment{example}[1][Example]{\begin{trivlist}
\item[\hskip \labelsep {\bfseries #1}]}{\end{trivlist}}
	\newenvironment{remark}[1][Remark]{\begin{trivlist}
\item[\hskip \labelsep {\bfseries #1}]}{\end{trivlist}}

	\newcommand{\qed}{\nobreak \ifvmode \relax \else
		      \ifdim\lastskip<1.5em \hskip-\lastskip
		            \hskip1.5em plus0em minus0.5em \fi \nobreak
			          \vrule height0.75em width0.5em depth0.25em\fi}
\newcommand{\githubi}{Git\textsc{h}ub}
\newcommand{\bdoh}{{\sc b}lack {\sc d}uck {\sc o}pen \textsc{hub}}
\newcommand{\ohloh}{\textsc{o}hloh}
\newcommand{\php}{\textsc{php}}
\newcommand{\twitter}{\textsc{t}witter}
\newcommand{\facebook}{\textsc{f}acebook}
\newcommand{\msn}{\textsc{msn}}
\newcommand{\gchat}{\textsc{g}oogle \textsc{c}hat}
\newcommand{\bash}{\textsc{b}ash}
\newcommand{\python}{\textsc{p}ython}
\newcommand{\django}{\textsc{d}jango}
\newcommand{\curl}{c\textsc{url}}
\newcommand{\firefox}{\textsc{f}irefox}
\newcommand{\floss}{\textsc{floss}}
\newcommand{\openoffice}{\textsc{o}pen\textsc{o}ffice}
\newcommand{\puredata}{\textsc{p}uredata}
\newcommand{\schema}{\textsc{s}chema.org}
\newcommand{\wiki}{\textsc{w}iki}
\newcommand{\nosql}{\textsc{n}o\textsc{sql}}
\newcommand{\etherpad}{\textsc{e}therpad}
\newcommand{\irc}{\textsc{irc}}
\newcommand{\irci}{\textsc{Irc}}
\newcommand{\ocd}{\textsc{ocd}}
\newcommand{\participa}{\textsc{p}articipa.br}
\newcommand{\httpb}{\textsc{http}}
\newcommand{\foaf}{\textsc{foaf}}
\newcommand{\ops}{\textsc{ops}}
\newcommand{\sioc}{\textsc{sioc}}
\newcommand{\gndo}{\textsc{gndo}}
\newcommand{\html}{\textsc{html}}
\newcommand{\ggg}{\textsc{ggg}}
\newcommand{\opa}{\textsc{opa}}
\newcommand{\obs}{\textsc{obs}}
\newcommand{\vbs}{\textsc{vbs}}
\newcommand{\lod}{\textsc{lod}}
\newcommand{\nlp}{\textsc{nlp}}
\newcommand{\sectionb}{\textsc{s}ection}
\newcommand{\cn}{\textsc{cn}}
\newcommand{\aab}{\textsc{aa}}
\newcommand{\dc}{\textsc{d}ublin {\sc c}ore}
\newcommand{\json}{\textsc{json}}
\newcommand{\flask}{\textsc{f}lask}
\newcommand{\aai}{\textsc{Aa}}
\newcommand{\ontologiaa}{\textsc{o}ntologi\textsc{aa}}
\newcommand{\ontologiaai}{\textsc{O}ntologi\textsc{aa}}
\newcommand{\owl}{{\sc owl}}
\newcommand{\www}{{\sc www}}
\newcommand{\rdfi}{{\sc Rdf}}
\newcommand{\mongodb}{{\sc m}ongo{\sc db}}
\newcommand{\mysql}{{\sc m}y{\sc sql}}
\newcommand{\rdf}{{\sc rdf}}
%\newcommand{\paaineli}{P{\sc aa}inel}
\newcommand{\paaineli}{P{\bf \sc aa}inel}
\newcommand{\paainel}{p{\sc aa}inel}
\newcommand{\gsd}{\textsc{gsd}}
\newcommand{\ui}{\textsc{ui}}
%\newcommand{\lmb}{\url{lab\textsc{M}acambira.sf.net}}
\newcommand{\lm}{lab\textsc{M}acambira.sf.net}
%\newcommand{\lm}{\url{labMacambira.sf.net}}



\makeatletter
\renewcommand\@biblabel[1]{\textbf{#1.}} % Change the square brackets for each bibliography item from '[1]' to '1.'
\renewcommand{\@listI}{\itemsep=0pt} % Reduce the space between items in the itemize and enumerate environments and the bibliography

\usepackage{epigraph}
%\pretocmd{\chapter}{\addtocontents{toc}{\protect\addvspace{5\p@}}}{}{}
%\pretocmd{\section}{\addtocontents{toc}{\protect\vspace{-4mm}}}{}{}
\renewcommand{\maketitle}{ % Customize the title - do not edit title and author name here, see the TITLE block below
\begin{flushright} % Right align
{\LARGE\@title} % Increase the font size of the title

\vspace{20pt} % Some vertical space between the title and author name

{\large\@author} % Author name
\\\@date % Date

\vspace{30pt} % Some vertical space between the author block and abstract
\end{flushright}
}

%----------------------------------------------------------------------------------------
%   TITLE
%----------------------------------------------------------------------------------------

\title{\textbf{A simple model that explains why inequality is inevitable and ubiquitous}\\ % Title
%a natural collective focus\\on the collective being} % Subtitle
} % Subtitle

\author{\textsc{Renato Fabbri, Osvaldo N. Oliveira Jr.} % Author
\\{\textit{São Carlos Institute of Physics, University of São Paulo, CP 369, 13560-970 São Carlos, SP, Brazil}}} % Institution

\date{\today} % Date

%----------------------------------------------------------------------------------------

\begin{document}

\maketitle % Print the title section

%----------------------------------------------------------------------------------------
%   ABSTRACT AND KEYWORDS
%----------------------------------------------------------------------------------------

%\renewcommand{\abstractname}{Summary} % Uncomment to change the name of the abstract to something else

%
\begin{abstract}
	Inequality has always been a crucial issue for human kind, particularly concerning the highly unequal distribution of wealth, which is at the root of major problems facing humanity, including extreme poverty and wars.
	A quantitative observation of inequality has become commonplace in recent years with the discovery that many natural as well as man-made systems can be represented as scale-free networks, whose distribution of connectivity obeys a power law.
	These networks may be generated by the preferential attachment for the nodes, within the so-called rich-gets-richer paradigm.
	In this letter we introduce a simple model that explains the ubiquity of inequality, based on three simple assumptions applied to a generic system.
	The first assumption is that the amount of each resource input to the system is fixed, as in a conservation law. Second assumption is the diversity of the components. The third assumption is an uniform distribution of resources along component wealth.
%	The first proposition is that there is diversity among components and not distinguish component cost in allocating resources.
	This implies that the more resources are allocated per component, less components with such cost the system presents, with the conservation of the amount of resources distributed through cost sweep.
	This can be geometrically described by the distribution of object sizes in a 3D space, where each dimension is assumed to be isotropic.
	Applying these assumptions to a generic system results in a power-law distribution, whose coefficient is the number of inputs that are independent from each other, i.e. the dimensionsionality of the allocated resources.
	Even though there is no restriction to the value of the coefficient, in practice we observe that existing systems normally exhibit a coefficient between 1.5 and 3.0.
	With our simple model it is not possible to determine whether this limitation in the coefficient values arises from a fundamental principle, but we indicate reasonable hypotheses.
	The assumptions in the model are analogous to the first and second laws of thermodynamics: conservation of resources and a time arrow pointing to inequality.
	Since these assumptions are easily justified based on established knowledge, the model proves unequivocally that inequality is ubiquitous.
	We also discuss ways to control this tendency to inequality, which is analogous to a decrease in entropy in a closed system induced with an external action.
\end{abstract}
%
%{
%\selectlanguage{portuguese}
%\begin{abstract}
%
%\end{abstract}
%}

\hspace*{3,6mm}\textit{Keywords:} power laws, fundamental theory, complex systems, complex networks, anthropological physics
%, statistics % Keywords

%\vspace{30pt} % Some vertical space between the abstract and first section

%----------------------------------------------------------------------------------------
%   ESSAY BODY
%----------------------------------------------------------------------------------------
%\newpage
%\tableofcontents
%\vspace*{1cm}
%{\bf This is a report on the newborn concept of \emph{anthropological physics}. Further efforts should contextualize, develop and correct theoretical nuances. The sharing of this naive text is a convenient step to the collective maturing and research.}
%\vspace*{.6cm}

%\newpage
%\epigraph{A single dramatic incident involving a breach of privacy could produce a set of statutes, rules, and prohibitions that could strangle the nascent field of computational social science in its crib. What is necessary, now, is to produce a self-regulatory regime of procedures, technologies, and rules that reduce this risk but preserve most of the research potential.}{David Lazer, Alex (Sandy) Pentland, Lada Adamic, Sinan Aral, Albert Laszlo Barabasi, Devon Brewer, Nicholas Christakis, Noshir Contractor, James Fowler, Myron Gutmann, Tony Jebara, Gary King, Michael Macy, Deb Roy, and Marshall Van Alstyne~\cite{life}}

\section{Introduction}
In this letter, we present inequality as resulting from
an  uniform distribution of resources with
respect to the resources already allocated to each component.


\subsection{Power laws such as Zipf and Pareto}
A power law is a functional relationship between two quantities $P(k)$ and $k$ in the form:
\begin{equation}
P(k) = Ck^{-\alpha}
\end{equation}
\noindent where $k\in [k_L,k_R]$
and $C$ is constant.
Assuming idealized phenomena:
\begin{equation}
C=\frac{1-\alpha}{k_R^{1-\alpha}-k_L^{1-\alpha}}
\end{equation}
\noindent Often, $k_R\rightarrow \infty$ which implies $\alpha>1$ as a condition for convergence of $P(k)$.
In such cases, the power law has a well-defined mean only if $\alpha>1$,
a finite variance only when $\alpha>2$. Well-defined skewness and kurtosis are restricted
to the cases where $\alpha>3$ and $\alpha>4$ respectively.

In nearly all systems, power laws are observed through both theory and empirical data.
Of special interest in the last decades are the scale-free complex networks,
the basic characteristic of which is a power law distribution of connectivity (number of edges per node).
Power laws also govern perception, as exposed by the Webner-Fechner and Stevens laws.
As a rule of thumb, the distribution of resources among (often self-interested) components
tends to follow a power law,
which includes distribution of human wealth, interactions, friendships;
connections among airports, synaptic count among neurons.
Some advocate 
about a better fit and theoretical backbone for the superposition of a
 power law distribution and a Weibull distribution~\cite{powWeib}.
 Most canonical examples in literature seem to be earthquake intensity and allometric relations of animal bodies,
 most canonical law examples seem to be Pareto and Zipf laws.
 Examples in basic physics are numerous, e.g. in a Newtonian context force is related to distance with $\alpha=2$ and force is related to acceleration with $\alpha=-1$.

\subsection{Related work}
Power laws. See at least~\cite{part,pbook}.

\section{Formalization}\label{sec:form}
\begin{definition}
	A resource-based system is a complex system that has an underlying resource vital to the their components and their interdependent roles in the system, often expressed as a power law.
\end{definition}

\subsection{Propositions and corollaries}
% corollaries, lemmas, etc
\begin{proposition}
	The amount of each resource input to the system is fixed.
\end{proposition}

\begin{proposition}
	There is diversity among the components of the resource-based system.
\end{proposition}

\begin{proposition}
	In allocating resources, a resource-based system does not distinguish the resources already allocated per component, expressed as an uniform distribution $p(k)$ of resources $k$.
\end{proposition}

\begin{corollary}
	The extension of allocation is $[0,\infty]$
\end{corollary}

%\begin{corollary}
%	For a system with a finite quantity of resources,
%	the allocation is compact with a superior limit $k_2$.
%\end{corollary}

\begin{corollary}
	The superior limit $k_2$ of the observed allocation of resources is an estimate of the amount of resources equally distributed along component wealth.
\end{corollary}

\begin{corollary}
	The dimensionality of the allocated resources is the scaling factor $\alpha$.
\end{corollary}




\subsection{Phenomenological approach}
% pure mathematical interpretation of the power law
Let k denote the amount of edges allocated by each vertex, its degree. We know that scale-free networks follows a power law distribution of degree p(k), in the form $p(k)=C/k^{-\alpha}$ with C a constant. Our understanding is that this distribution stems from an equanimous distribution of resources in $\alpha$ dimensions, in a manner that resembles wave theory.

For a moment, regard $p(k)$ as the frequency of occurance $f$. Regard $k$ as the amount of resource, now the time period $\lambda$. If there is no change of medium, the wave travels at a constant rate $v=\lambda . f$. Recal $p(k) = C / k^\alpha$. Accordingly, we can observe $f = v / \lambda$ and that, in this case, $v = C$ and $\alpha = 1$.

Now let $f=v/(\lambda_1 . \lambda_2)$, that is, the frequency of occurance (or the probability of choosing such event at random) go with the inverse of two periods. 




\section{Paradigmatic examples}
\subsection{Workers in a factory}
Let us consider a generic problem in which a System ($S$) provides an Output ($O$) depending on the Input ($I$) it receives. The following assumptions are established.
1) $S$ is made of a number of components that are not all equal to each other. That is to say, there is diversity in the nature of the components. 
2) There may be several inputs, but for each input the amount of resources furnished to the System is fixed, as in a conservation law. 
3) Distribution should be uniform with regard to the “size” of the component as in a geometric case where space is considered as isotropic. 

There is no assumption for the Output (O), which is taken as to mean the performance (or richness) in terms of the components of S. 
Now assuming that there are N types of input, and for the sake of the argument, all of them have a time dependence (with 1/t), according to assumption 2) above. The Output is the product of the functions of these N inputs. 
O = (R1*R2*… RN)/tN since a given input can be written as Ri/t. 
The Output has therefore a power-law dependence on t with coefficient N. 
Now considering the values of N observed in practice (from many examples of power-law dependences), which is normally between 1.5 (2?) and 3, one infers that there are at least two types (?) of independent inputs and at maximum 3 independent inputs. 
Let us illustrate with a hypothetical case that may facilitate understanding the concepts. A piece of work is to be done in a company. What sort of resources can be established as inputs? We assume three inputs: number of workers, working hours and efficiency. We recall that all resources should be fixed and that there is diversity in the components. 
Then, first the total number of workers available are divided into groups of different sizes,
Continuar exemplo ????

Falta resumir a literatura que mostra coeficiente entre 1.5 e 3. Mencionar casos em que é maior que 3.
Incluir exemplos em que a distribuição uniforme se dá, mesmo que sejam empíricos. Lembro que você tinha isso para lista de e-mails, e acho que outros exemplos com maior apelo para a física precisariam ser incluídos.


\subsection{Object sizes in your house or elsewhere}

Que estah interessado na sua casa em objetos de dois tamanhos: um quase do tamanho de um cubo unitario (seja qual for a unidade q vc escolher); e outro quase do tamanho l (l inteiro e maior q um). Em cada cubo unitario ha rho de chance de ter um objeto com o tamanho almejado. Nos cubos  de lado l tb. Dado um cubo de lado n (l divide n) ha, em media, rho . $(n/l)^3$ objetos de volume $~l^3$ e $\rho . n^3$ objetos de volume $~1^3$. Ou seja, a frequencia de objetos com volume X eh inversamente proporcional ao volume X.
\section{Especial cases}
The consequences of the
interpretation of power laws presented in Section~\ref{sec:form}
are severe for understanding and dealing with
phenomena. We expose selected cases in this section.


\subsection{Scale free complex networks}
We advocate that this is the similar case of that where edges reflect the resources allocated by individuals. If $f=v/resources_i=v/E_i=v/(N_i . (E_i/N_i))=2v/(N_i . k_i) \equiv v / (\lambda_1 .   \lambda_2)$. As $N_i$ and $k_i$ are both directly proportional to $E_i$, which is the primary resource, one can factor out another constant and consider the special case where $\lambda=\lambda_1=\lambda_2$ and, consequently, $f=v/\lambda^2$. E.g. in a social network, the number of agents allocated and the time each of them put, are seen as the primary resource (individual . time).

Questions:
*) the range of degree covered by scale-free networks is maximum, as do our perception, which also follows power laws. How far can we consider scale-free complex networks to be meta-sensors that captures and processes signals about the very reason of existance of the meta-sensor?

Theorem 1: every scale-free network with distribution of degree $p(k)=C/k^\alpha$ can be understood as having an equanimous distribution of resources in $\alpha$ dimensions.
Corolary: if $p(E_i)=C/E_i$, with $\alpha=1$. (might have to consider only edges with vertices of other connectivity, i.e. discard edges between vertices with same degree.)

\subsection{Meta-sensors}
Perception presents many psychophysical power-law 
relations between
magnitude of the physical stimulus and the perceived 
(subjective) quantity~\cite{stev,web}.
This is usually attributed to the utility of perception capability,
which is considerably enhanced upon broadening of the spectrum.
Another explanation is on the physical phenomena itself.
Consider a sound wave traveling with constant speed $v$.
If the organism will consider wave lengths from $\lambda_1$
to $\lambda_2$, $f=\frac{v}{\lambda} \in [\frac{}]$ follows
a power law with $\alpha=1$.

In either case, the persistence of power laws in perception
suggests a pertinence, and is regarded as such. This raises
points a fit, in advance, for complex systems with power laws
to be thought of as sensor (or meta-sensors) on signals
of the domain of the resource $k$.
For example, an interest group on hiking can be understood as
a meta-sensor about hiking and involved community: current good
places, equipment, people, proper behavior, etc.
The power law, i.e. ``the scale-free trace'' to use
the complex network jargon, sweeps the broadest 
diversity of engagement, which can be regarded as
inversely proportional the diversity brought to the group
by the participant~\cite{tStable}: 
as one allocates more resources (say time)
in one system (or a set of them),
it allocates less resources is the rest of the systems.

\subsection{Equanimous inequality}
Paradoxically, power laws, which is the 
current utmost inequality paradigm,
follow from an equanimous consideration
of resources and exhibit other equanimous aspects:
\begin{itemize}
	\item $p(k)=C.k^{-\alpha} \Rightarrow p(k).k^{\alpha}=C$, with C constant. That is, the amount $k$ of resources per component times the amount of those components, which is the total ``instantaneous'' allocated resources, is constant $C$. The scaling factor $\alpha$ is herein interpreted as the number of dimensions in which such resources are being observed.
	\item Each component participates in numerous other complex systems, potentially infinite, and should present a broad, if not complete, sweep of resources allocated to itself.
		These resources are not necessarily of the same type. We assume that human systems, for example, present power-law distributions of knowledge $p_k(k_k)$ and of wealth $p_w(k_w)$, with potentially different (relative) amount $k$ of resources. At the same time, within a fixed type of resource, resources allocated vary in  different systems. For example, an individual tends to have many acquaintances (fixed resource) in its own family, work and neighborhood, a fewer knowns in such circles of distant family members, partners and friends.
	\item The distinction of each component particularities is often not of core importance to describe complex behavior.
		This reflects in symmetries among components. For example, Human individuals form complex social systems with power law distributions of relations. All the participants, by being humans, have the same amount of time available each day, resource, to engage in all the complex systems that are presented by the environment. 
\end{itemize}


\subsection{Wealth distribution}
One manifestation of the power law
which is most fundamental to daily experience
in current society is the inequality of wealth distribution.
There are continuous efforts to deal with this issue,
usually advocating ways to minimize ``social inequality''.
Considering the framework presented withing this letter:

\begin{itemize}
	\item such an inequality is a natural tendency.

	\item This should make work to diminish it

	\item not any unequal outline, but a power law.
\end{itemize}

In particular, the (publicized)
homogeneity of earnings in public institutions,
and the (publicized) distribution of wealth in whole countries,
reveal that there is indeed efforts to minimize
the strong inequality imposed by power laws.
Additionally, publicized data should be regarded with
extra care and scepticism, as they do not present the
expected power-law distributions.




\subsection{Naturalization of inequality}
% proposta de recursos = E ~ k . N para a lei de potência em sistemas reais

\section{Conclusions}
\subsection*{Acknowledgments}
Authors thank
the General Secretariat of the Republic Presidency (SG-PR) and UNDP for supporting this
research (contract 2013/00056, project BRA/12/018); the National Counsel of Technological 
and Scientific Development (process 140860/2013-4, project 870336/1997-5,
advisor: 
all
labMacambira.sf.net members for pursuing this and other developments;
%----------------------------------------------------------------------------------------
%   BIBLIOGRAPHY
%----------------------------------------------------------------------------------------

%\bibliographystyle{unsrt}
%\bibliographystyle{plain}
\bibliographystyle{ieeetr}
\bibliography{essay}

%----------------------------------------------------------------------------------------

\end{document}
