%%%%%%%%%%%%%%%%%%%%%%%%%%%%%%%%%%%%%%%%%
% Thin Sectioned Essay
% LaTeX Template
% Version 1.0 (3/8/13)
%
% This template has been downloaded from:
% http://www.LaTeXTemplates.com
%
% Original Author:
% Nicolas Diaz (nsdiaz@uc.cl) with extensive modifications by:
% Vel (vel@latextemplates.com)
%
% License:
% CC BY-NC-SA 3.0 (http://creativecommons.org/licenses/by-nc-sa/3.0/)
%
%%%%%%%%%%%%%%%%%%%%%%%%%%%%%%%%%%%%%%%%%

%----------------------------------------------------------------------------------------
%   PACKAGES AND OTHER DOCUMENT CONFIGURATIONS
%----------------------------------------------------------------------------------------

\documentclass[a4paper, 11pt]{article} % Font size (can be 10pt, 11pt or 12pt) and paper size (remove a4paper for US letter paper)
\usepackage{hyperref}
\hypersetup{
        colorlinks,
            linkcolor={red!50!black},
                citecolor={blue!50!black},
                    urlcolor={blue!80!black}
                }


\usepackage[portuguese,english]{babel}
\usepackage[utf8]{inputenc}
%\usepackage{float}
\DeclareUnicodeCharacter{202D}{}
\DeclareUnicodeCharacter{202C}{}

%\usepackage{color} % for the notes
\usepackage{xcolor}
\usepackage[protrusion=true,expansion=true]{microtype} % Better typography
%\usepackage{graphicx} % Required for including pictures
%\usepackage{wrapfig} % Allows in-line images
%\usepackage{tocloft}
%\usepackage{multirow}

\usepackage{mathpazo} % Use the Palatino font
%\usepackage[T1]{fontenc} % Required for accented characters
%\linespread{1.05} % Change line spacing here, Palatino benefits from a slight increase by default
%\usepackage{etoolbox}

%\newtheorem{theorem}{Theorem}[section]
\newtheorem{theorem}{Theorem}
\newtheorem{theorem2}{Theorem}
\newtheorem{theorem3}{Theorem}
\newtheorem{lemma}[theorem]{Lemma}
\newtheorem{axiom}[theorem]{Axiom}
\newtheorem{proposition}[theorem]{Proposition}
\newtheorem{corollary}[theorem2]{Corollary}
\newtheorem{definition2}[theorem3]{Definition}


\newenvironment{proof}[1][Proof]{\begin{trivlist}
\item[\hskip \labelsep {\bfseries #1}]}{\end{trivlist}}
	\newenvironment{definition}[1][Definition]{\begin{trivlist}
\item[\hskip \labelsep {\bfseries #1}]}{\end{trivlist}}
	\newenvironment{example}[1][Example]{\begin{trivlist}
\item[\hskip \labelsep {\bfseries #1}]}{\end{trivlist}}
	\newenvironment{remark}[1][Remark]{\begin{trivlist}
\item[\hskip \labelsep {\bfseries #1}]}{\end{trivlist}}

	\newcommand{\qed}{\nobreak \ifvmode \relax \else
		      \ifdim\lastskip<1.5em \hskip-\lastskip
		            \hskip1.5em plus0em minus0.5em \fi \nobreak
			          \vrule height0.75em width0.5em depth0.25em\fi}
\newcommand{\githubi}{Git\textsc{h}ub}
\newcommand{\bdoh}{{\sc b}lack {\sc d}uck {\sc o}pen \textsc{hub}}
\newcommand{\ohloh}{\textsc{o}hloh}
\newcommand{\php}{\textsc{php}}
\newcommand{\twitter}{\textsc{t}witter}
\newcommand{\facebook}{\textsc{f}acebook}
\newcommand{\msn}{\textsc{msn}}
\newcommand{\gchat}{\textsc{g}oogle \textsc{c}hat}
\newcommand{\bash}{\textsc{b}ash}
\newcommand{\python}{\textsc{p}ython}
\newcommand{\django}{\textsc{d}jango}
\newcommand{\curl}{c\textsc{url}}
\newcommand{\firefox}{\textsc{f}irefox}
\newcommand{\floss}{\textsc{floss}}
\newcommand{\openoffice}{\textsc{o}pen\textsc{o}ffice}
\newcommand{\puredata}{\textsc{p}uredata}
\newcommand{\schema}{\textsc{s}chema.org}
\newcommand{\wiki}{\textsc{w}iki}
\newcommand{\nosql}{\textsc{n}o\textsc{sql}}
\newcommand{\etherpad}{\textsc{e}therpad}
\newcommand{\irc}{\textsc{irc}}
\newcommand{\irci}{\textsc{Irc}}
\newcommand{\ocd}{\textsc{ocd}}
\newcommand{\participa}{\textsc{p}articipa.br}
\newcommand{\httpb}{\textsc{http}}
\newcommand{\foaf}{\textsc{foaf}}
\newcommand{\ops}{\textsc{ops}}
\newcommand{\sioc}{\textsc{sioc}}
\newcommand{\gndo}{\textsc{gndo}}
\newcommand{\html}{\textsc{html}}
\newcommand{\ggg}{\textsc{ggg}}
\newcommand{\opa}{\textsc{opa}}
\newcommand{\obs}{\textsc{obs}}
\newcommand{\vbs}{\textsc{vbs}}
\newcommand{\lod}{\textsc{lod}}
\newcommand{\nlp}{\textsc{nlp}}
\newcommand{\sectionb}{\textsc{s}ection}
\newcommand{\cn}{\textsc{cn}}
\newcommand{\aab}{\textsc{aa}}
\newcommand{\dc}{\textsc{d}ublin {\sc c}ore}
\newcommand{\json}{\textsc{json}}
\newcommand{\flask}{\textsc{f}lask}
\newcommand{\aai}{\textsc{Aa}}
\newcommand{\ontologiaa}{\textsc{o}ntologi\textsc{aa}}
\newcommand{\ontologiaai}{\textsc{O}ntologi\textsc{aa}}
\newcommand{\owl}{{\sc owl}}
\newcommand{\www}{{\sc www}}
\newcommand{\rdfi}{{\sc Rdf}}
\newcommand{\mongodb}{{\sc m}ongo{\sc db}}
\newcommand{\mysql}{{\sc m}y{\sc sql}}
\newcommand{\rdf}{{\sc rdf}}
%\newcommand{\paaineli}{P{\sc aa}inel}
\newcommand{\paaineli}{P{\bf \sc aa}inel}
\newcommand{\paainel}{p{\sc aa}inel}
\newcommand{\gsd}{\textsc{gsd}}
\newcommand{\ui}{\textsc{ui}}
%\newcommand{\lmb}{\url{lab\textsc{M}acambira.sf.net}}
\newcommand{\lm}{lab\textsc{M}acambira.sf.net}
%\newcommand{\lm}{\url{labMacambira.sf.net}}



\makeatletter
\renewcommand\@biblabel[1]{\textbf{#1.}} % Change the square brackets for each bibliography item from '[1]' to '1.'
\renewcommand{\@listI}{\itemsep=0pt} % Reduce the space between items in the itemize and enumerate environments and the bibliography

\usepackage{epigraph}
%\pretocmd{\chapter}{\addtocontents{toc}{\protect\addvspace{5\p@}}}{}{}
%\pretocmd{\section}{\addtocontents{toc}{\protect\vspace{-4mm}}}{}{}
\renewcommand{\maketitle}{ % Customize the title - do not edit title and author name here, see the TITLE block below
\begin{flushright} % Right align
{\LARGE\@title} % Increase the font size of the title

\vspace{20pt} % Some vertical space between the title and author name

{\large\@author} % Author name
\\\@date % Date

\vspace{30pt} % Some vertical space between the author block and abstract
\end{flushright}
}

%----------------------------------------------------------------------------------------
%   TITLE
%----------------------------------------------------------------------------------------

\title{\textbf{A simple model that explains why inequality is inevitable and ubiquitous}\\ % Title
%a natural collective focus\\on the collective being} % Subtitle
} % Subtitle

\author{\textsc{Renato Fabbri, Osvaldo N. Oliveira Jr.} % Author
\\{\textit{São Carlos Institute of Physics, University of São Paulo, CP 369, 13560-970 São Carlos, SP, Brazil}}} % Institution

\date{\today} % Date

%----------------------------------------------------------------------------------------

\begin{document}

\maketitle % Print the title section

%----------------------------------------------------------------------------------------
%   ABSTRACT AND KEYWORDS
%----------------------------------------------------------------------------------------

%\renewcommand{\abstractname}{Summary} % Uncomment to change the name of the abstract to something else

%
\begin{abstract}
	Inequality has always been a crucial issue for human kind, particularly concerning the highly unequal distribution of wealth, which is at the root of major problems facing humanity, including extreme poverty and wars.
	A quantitative observation of inequality has become commonplace in recent years with the discovery that many natural as well as man-made systems can be represented as scale-free networks, whose distribution of connectivity obeys a power law.
	These networks may be generated by the preferential attachment for the nodes, within the so-called rich-gets-richer paradigm.
	In this letter we introduce a simple model that explains the ubiquity of inequality, based on three simple assumptions applied to a generic system.
	The first assumption is that the amount of each resource input to the system is fixed, as in a conservation law. Second assumption is the diversity of the components. The third assumption is an uniform distribution of resources along component wealth.
%	The first proposition is that there is diversity among components and not distinguish component cost in allocating resources.
	This implies that the more resources are allocated per component, less components with such cost the system presents, with the conservation of the amount of resources distributed through cost sweep.
	This can be geometrically described by the distribution of object sizes in a 3D space, where each dimension is assumed to be isotropic.
	Applying these assumptions to a generic system results in a power-law distribution, whose coefficient is the number of inputs that are independent from each other, i.e. the dimensionsionality of the allocated resources.
	Even though there is no restriction to the value of the coefficient, in practice we observe that existing systems normally exhibit a coefficient between 1.5 and 3.0.
	With our simple model it is not possible to determine whether this limitation in the coefficient values arises from a fundamental principle, but we indicate reasonable hypotheses.
	The assumptions in the model are analogous to the first and second laws of thermodynamics: conservation of resources and a time arrow pointing to inequality.
	Since these assumptions are easily justified based on established knowledge, the model proves unequivocally that inequality is ubiquitous.
	We also discuss ways to control this tendency to inequality, which is analogous to a decrease in entropy in a closed system induced with an external action.
\end{abstract}
%
%{
%\selectlanguage{portuguese}
%\begin{abstract}
%
%\end{abstract}
%}

\hspace*{3,6mm}\textit{Keywords:} power laws, fundamental theory, complex systems, complex networks, anthropological physics
%, statistics % Keywords

%\vspace{30pt} % Some vertical space between the abstract and first section

%----------------------------------------------------------------------------------------
%   ESSAY BODY
%----------------------------------------------------------------------------------------
%\newpage
%\tableofcontents
%\vspace*{1cm}
%{\bf This is a report on the newborn concept of \emph{anthropological physics}. Further efforts should contextualize, develop and correct theoretical nuances. The sharing of this naive text is a convenient step to the collective maturing and research.}
%\vspace*{.6cm}

%\newpage
%\epigraph{A single dramatic incident involving a breach of privacy could produce a set of statutes, rules, and prohibitions that could strangle the nascent field of computational social science in its crib. What is necessary, now, is to produce a self-regulatory regime of procedures, technologies, and rules that reduce this risk but preserve most of the research potential.}{David Lazer, Alex (Sandy) Pentland, Lada Adamic, Sinan Aral, Albert Laszlo Barabasi, Devon Brewer, Nicholas Christakis, Noshir Contractor, James Fowler, Myron Gutmann, Tony Jebara, Gary King, Michael Macy, Deb Roy, and Marshall Van Alstyne~\cite{life}}

\section{Introduction}
Symmetry, and its symmetric twin, asymmetry, or inequality,
is regarded as fundamental to cognition and the universal laws
in nearly all fields, emanating from physics and philosophy.
The ubiquity of symmetry is appreciated so profoundly that
literature constantly recapitulates that the human mind present
explicit and basic symmetry operations through thinking.
Being the way our mind works, we model the world with symmetry
regardless if the world present those symmetries.
Models are useful, and the symmetry observed through scientific
means are believed to reflect reality, but it is also foreseen
that new knowledge should spout from a somewhat different paradigm.

In this letter, we present inequality as resulting from
an uniform distribution of resources with
respect to the resources already allocated to each component.


\subsection{Power laws such as Zipf and Pareto}
A power law is a functional relationship between two quantities $P(k)$ and $k$ in the form:
\begin{equation}\label{eq:pow}
P(k) = Ck^{-\alpha}
\end{equation}
\noindent where $k\in [k_L,k_R]$
and $C$ is constant.
Assuming idealized phenomena:
\begin{equation}
C=\frac{1-\alpha}{k_R^{1-\alpha}-k_L^{1-\alpha}}
\end{equation}
\noindent Often, $k_R\rightarrow \infty$ which implies $\alpha>1$ as a condition for convergence of $P(k)$.
In such cases, the power law has a well-defined mean only if $\alpha>1$,
a finite variance only when $\alpha>2$. Well-defined skewness and kurtosis are restricted
to the cases where $\alpha>3$ and $\alpha>4$ respectively.

In nearly all systems, power laws are observed through both theory and empirical data.
Of special interest in the last decades are the scale-free complex networks,
the basic characteristic of which is a power law distribution of connectivity (number of edges per node).
Power laws also govern perception, as exposed by the Webner-Fechner and Stevens laws.
As a rule of thumb, the distribution of resources among (often self-interested) components
tends to follow a power law,
which includes distribution of human wealth, interactions, friendships;
connections among airports, synaptic count among neurons.
Some advocate 
about a better fit and theoretical backbone for the superposition of a
 power law distribution and a Weibull distribution~\cite{powWeib}.
 Most canonical examples in literature seem to be earthquake intensity and allometric relations of animal bodies,
 most canonical law examples seem to be Pareto and Zipf laws.
 Examples in basic physics are numerous, e.g. in a Newtonian context force is related to distance with $\alpha=2$ and force is related to acceleration with $\alpha=-1$.

\subsection{Related work}
Power laws. See at least~\cite{part,pbook}.

\section{Formalization}\label{sec:form}

\begin{definition2}
	A {\bf resource} is anything that is used by a complex system to subsist and communicate. The term is used herein for a resource in evidence for the persistence of the complex system considered.
\end{definition2}

 Usually, the complex system is located roughly through its components and the resources that keep the system altogether.

\begin{definition2}
	A {\bf resource-based system} is a complex system that has an underlying resource vital to the their components and their interdependent roles in the system.
\end{definition2}

Specially in Game Theory, these components are often considered
``self-interested''. This is not part of our definition and
is not a required condition for the framework here presented.
Even so, we understand that this formalization shed insight
in the reasons and way that self-interested agents
organize themselves
with extensive incidence of power-laws.

\begin{definition2}
	The {\bf component wealth} $k$ is the amount of resources allocated to the component.
\end{definition2}

We will use $p(k)$ to denote the fraction of components with component wealth $k$. Likewise, the same notation $p(k)$ will denote the probability of choosing a component with an amount $k$ of resources.
The context should make it clear which is the appropriate meaning.
This symmetry among probability, frequency and relative count is
instrumental for the interpretative framework herein presented,
which is conveniently embedded in the notation.
It is also useful to observe 
$k$ as $\lambda$, i.e. as a wave period.
Then, $p(k)$ is understood as  $f$, the
frequency of occurrence.

\subsection{Propositions and corollaries}
% corollaries, lemmas, etc

Inspired in the laws of thermodynamics, we derived four
propositions.
These principles can be though of as laws met in
a very broad class of phenomena.
These propositions follow from the complexity of the system:
there is so much involved, that the ignorance is assured and
one needs to grasp hardly false assumptions. 
This leads to:


\begin{proposition}\label{prop:0}
	There is diversity among the components of the resource-based system. (Zeroth law)
\end{proposition}

If there is a big set of components, there is hardly any way to avoid diversity. Be it location, size, age, the way someone regards them, etc., distinctions arise (as do symmetry). This can be regarded as both a statistical law, or even as a deeper truth.
The mere existence of two objects imposes diversity,
otherwise they would be both the same.

Proposition~\ref{prop:0} is required for the attribution of different amounts of resources for each component:
if they are equal, by definition they have the same component wealth.

\begin{proposition}\label{prop:1}
	The amount of each resource input to the system is fixed. (First law)
\end{proposition}

This follows from the independence of each resource dimension,
say $\lambda_1$ and $\lambda_2$, and
the multiplicative relation they hold with the total resource $E=\lambda_1 . \lambda_2$. Next section holds this discussion more throughly.

\begin{proposition}\label{prop:2}
	In allocating resources, a resource-based system does not distinguish the resources already allocated in the components. This is expressed as an uniform distribution $p_U(k)$ of resources with respect to component wealth $k$. (Second law)
\end{proposition}

That is, resources are allocated without distinction to component wealth, which has two major consequences: 1) diversity of component wealth is maximized; 2) the wealthier the components considered, the fewer they are. 

\begin{proposition}\label{prop:3}
	The ground state implied by the everlasting validity of the second law is characterized by
	a power-law distribution $p(k)$ of components with component wealth $k$. Deviations from $p(k)$ tend to be transient or require effort, the expenditure of energy (as work in needed to reduce entropy), or be imposed by harsh conditions (such as an apple that does not fall if stuck in the ceiling). (Third law.)
\end{proposition}

This follows from the second law and the definition of $k$:
$p(k)=\frac{p_U(k)}{k^\alpha}$, where $\alpha$ is
the dimensionality of the resource.
As the system continues to exist,
and resources are continually allocated, deviations from the equanimous
distribution $p_U(k)$ of resources along component wealth $k$,
 tend to be transient.
 Notice that $p_U(k)$ is usually not observed, but only through 
 the power-law distribution $p(k)$ of components
with component wealth $k$.

\begin{corollary}
	The extension of allocation is $[k_L,k_R]$ with $R_L$ often 0 or 1 and $k_R\approx C$.
\end{corollary}

This follows from Proposition~\ref{prop:2}.
If the allocation of resources is insensitive to component wealth,
it should sweep all possible values, and these are usually
bounded bellow by being a positive quantity of resources.
Most usually, systems are considered as a set of components
or the components in which certain resources couple them,
in which $k_L=0$ and $K_R=1$ are reasonable, respectively.

The distributions $p(k)$ and $p_U(k)$ are also
bounded above when component wealth reaches $k_R \approx C$,
the amount of resources uniformly distributed along component wealth.
In empirical data, $k_R$ can vary considerably, due to
nonlinearity of the resources scaling (most often $k_R<C$) and
to self-interested agents (most often $k_R>C$).
Power laws are reported to conduct empirical data sovereignly and
paradoxically they most often are strict only for a
(broad) portion of component wealth range.

%\begin{corollary}
%	For a system with a finite quantity of resources,
%	the allocation is compact with a superior limit $k_2$.
%\end{corollary}

\begin{corollary}
	The superior limit $k_R$ of the observed allocation of resources is an estimate of the amount of resources $C$ equally distributed along component wealth ($C\approx k_2$).
\end{corollary}

\begin{corollary}
	$p(1)$ is another estimate of the amount of resources $C$ equally distributed along component wealth ($C\approx p(1)$).
\end{corollary}

\begin{corollary}
	An estimate for $C$ can also be found by $\alpha$, $k_L$ and $k_R$ through Equation~\ref{eq:pow}.
\end{corollary}

\begin{corollary}
	The dimensionality of the allocated resources is the scaling factor $\alpha$.
\end{corollary}

This last corollary follows from box counting or, most easily,
through wave-like reasoning about power laws,
the excursion of the next section.

\subsection{Phenomenological approach}
% pure mathematical interpretation of the power law

We shall also
doubt above definitions, propositions and corollaries 
and depart from a more phenomenological standpoint,
such as data and descriptive models.
Suppose there is a power law relation in empirical data
or driven from specific domain theory.
Is the conceptualization
presented above axiomatically still helpful?
We provide here a general and mathematically grounded
interpretation of any power law relationship,
perfectly consonant with such assumptions.
We understand that this short theoretical consideration
suggests, sustains and deepen them.
Both approaches sustain a quasi-``if and only if''
interpretation of power laws.

Consider a constant speed $v$ for a wave propagation
(suppose linear media with no dispersion).
Recall that the number of oscillations $f$ per unit time is
inversely proportional to the cycle length $\lambda$ (the period).
In usual notation $f=\frac{v}{\lambda}$.
The constant speed $v$ implies a power law between 
$f$ and $\lambda$ (with $\alpha=1$ and $C=v$).
This is a core insight, the general case of a power law can
be interpreted as resulting from a constant amount $C$ of
fundamental resources with dimensionality $\alpha$ 
being homogeneously distributed across
component wealth $k$ and 
resulting in $p(k)$ of such components.

Now let $f=\frac{v=C}{\lambda_1 . \lambda_2}$, that is,
the frequency of occurrence
(or the probability of choosing such event at random) go with the inverse of two periods while the speed is constant. 
If $\lambda_1==\lambda_2==\lambda$, then $f=\frac{v}{\lambda^2}$
In other words, the density given by an amount $v$
(or a quantity $C$ of resources) in a hypercube of
edge $\lambda$ (or component wealth $k$)
and $\alpha$ dimensions.
This same reasoning yields the amount, fraction or probability $p(k)$
of components with such volume.
Be $C_i$ the amount of resources allocated at
specific resource costs,
say $\lambda_{1,i}$ and $\lambda_{2,i}$ to strengthen the wave argumentation,
and assume linearity $\lambda_{1,i}=c.\lambda_{2,1}$,
such that
$p(k_i)=\frac{C}{C_i}=\frac{C}{\lambda_{1,i}.\lambda_{2,i}}=
\frac{C}{c\lambda_{2,i}^2}=\frac{\widetilde{C}}{\lambda_{2,i}^2}\equiv\frac{v}{\lambda^2}$.
Therefore one can consider that we only access $\lambda_1==\lambda_2$ and a ``normalized resource C'' allocated by the environment uniformly across component wealth: the higher component wealth implies less numerous components. 
All sorts of nonlinearity should account
for many kinds of deviations in empirical power laws.

Such interpretation holds for 
both wave and probabilistic phenomena.
This suggests that further mathematical
parallels might be useful in understanding complex
systems, be it through thermodynamic, wave or quantum theories.
It is not yet clear to which extent does this correspondence hold true,
but one might glimpse diverse severe hypothesis,
such as a lower bound for the energy involved in the integration of
the components into a complex system.
Would such energy account for part of the phenomena
currently explained through dark matter assumptions?

\section{Paradigmatic examples}
\subsection{Object sizes in your house or elsewhere}
Pick your size (or the size of your hand, your arm)
as a measure unit $l$ for length, pick $m \in (0,1)$ 
and $dm$ arbitrarily small.
In your house, there probably are more objects
of volume $\approx (m.l)^3$ (or that fit such volume)
than those of volume $\approx l^3$.
Furthermore, as we know nothing about your house,
we can assume that the chance $\rho$ of finding and object
fitting an arbitrary cube of volume $\approx (m.l)^3$
is the same as finding an object fitting an
arbitrary cube of volume of volume $\approx l^3$.
As your house has a fixed volume, there are $m^3$ more of
the smaller cubes and therefore $m^3$ more objects of such
volume. 
The result is a power-law distribution of object volumes
related to the length $l$ with $\alpha=3$:
$p(l)=C.l^-3$.
Notice that if the objects are mutually exclusive,
$\alpha$ is probably lower, as the number of smaller
cubes for decrease considerably.

This example holds a geometrical
interpretation of the formalism presented in the previous section.
It also might be assumed true even if there is no isotropy
and can be taken as a ``best guess'' if total ignorance
about the system is assumed. 
Extreme choices of $l$ and $m$ (and $dl$, $dm=mdl$ for tolerance in size)
will often exhibit a different behavior (e.g. $l=1$ light year for couting objects in a house) and, in this context, should be taken as an indicative that there is no complex system in the scales observed. 

\subsection{Workers in a factory}
Let us consider a generic problem in which a System ($S$) provides an Output ($O$) depending on the Input ($I$) it receives. The following assumptions are established.
1) $S$ is made of a number of components that are not all equal to each other. That is to say, there is diversity in the nature of the components. 
2) There may be several inputs, but for each input the amount of resources furnished to the System is the same, as in a conservation law. 
3) Distribution should be uniform with regard to the “size” of the component (component wealth) as in the geometric isotropic case of your house. 

There is no assumption for the Output ($O$), which is taken as to mean the performance (or richness) in terms of the components of $S$.
Now, assuming that there are $N$ types of input, and for the sake of the argument, all of them have a time dependence (with 1/t), according to assumption 2) above.
The Output is the product of the functions of these N inputs. 
O = (R1*R2*… RN)/tN since a given input can be written as Ri/t. 
The Output has therefore a power-law dependence on t with coefficient N. 
Now considering the values of N observed in practice (from many examples of power-law dependences), which is normally between 1.5 (2?) and 3, one infers that there are at least two types (?) of independent inputs and at maximum 3 independent inputs. 
Let us illustrate with a hypothetical case that may facilitate understanding the concepts.
A piece of work is to be done in a company.
What sort of resources can be established as inputs? We assume three inputs: number of workers, working hours and efficiency.
We recall that all resources should be fixed and that there is diversity in the components. 
Then, first the total number of workers available are divided into groups of different sizes,
%Continuar exemplo ????
%
%Falta resumir a literatura que mostra coeficiente entre 1.5 e 3. Mencionar casos em que é maior que 3.
%Incluir exemplos em que a distribuição uniforme se dá, mesmo que sejam empíricos. Lembro que você tinha isso para lista de e-mails, e acho que outros exemplos com maior apelo para a física precisariam ser incluídos.


\section{Especial cases}
The consequences of the
interpretation of power laws presented in Section~\ref{sec:form}
are severe for understanding and dealing with
phenomena. We expose selected cases in this section.


\subsection{Scale free complex networks}
We advocate that this is the similar case of that where edges reflect the resources allocated by individuals. If $f=v/resources_i=v/E_i=v/(N_i . (E_i/N_i))=2v/(N_i . k_i) \equiv v / (\lambda_1 .   \lambda_2)$. As $N_i$ and $k_i$ are both directly proportional to $E_i$, which is the primary resource, one can factor out another constant and consider the special case where $\lambda=\lambda_1=\lambda_2$ and, consequently, $f=v/\lambda^2$. E.g. in a social network, the number of agents allocated and the time each of them put, are seen as the primary resource (individual . time).

Questions:
*) the range of degree covered by scale-free networks is maximum, as do our perception, which also follows power laws. How far can we consider scale-free complex networks to be meta-sensors that captures and processes signals about the very reason of existance of the meta-sensor?

Theorem 1: every scale-free network with distribution of degree $p(k)=C/k^\alpha$ can be understood as having an equanimous distribution of resources in $\alpha$ dimensions.
Corolary: if $p(E_i)=C/E_i$, with $\alpha=1$. (might have to consider only edges with vertices of other connectivity, i.e. discard edges between vertices with same degree.)

\subsection{Meta-sensors}
Perception presents many psychophysical power-law 
relations between
magnitude of the physical stimulus and the perceived 
(subjective) quantity~\cite{stev,web}.
This is usually attributed to the utility of perception capability,
which is enhanced upon broadening of the spectrum.
Another explanation is on the physical phenomena itself.
Consider a sound wave traveling with constant speed $v$.
If the organism will consider wave lengths from $\lambda_1$
to $\lambda_2$, $f=\frac{v}{\lambda} \in [\frac{}]$ follows
a power law with $\alpha=1$.

In either case, the persistence of power laws in perception
suggests a pertinence, and is regarded as such. This raises
points a fit, in advance, for complex systems with power laws
to be thought of as sensor (or meta-sensors) on signals
of the domain of the resource $k$.
For example, an interest group on hiking can be understood as
a meta-sensor about hiking and involved community: current good
places, equipment, people, proper behavior, etc.
The power law, i.e. ``the scale-free trace'' to use
the complex network jargon, sweeps the broadest 
diversity of engagement, which can be regarded as
inversely proportional the diversity brought to the group
by the participant~\cite{tStable}: 
as one allocates more resources (say time)
in one system (or a set of them),
it allocates less resources is the rest of the systems.

\subsection{Equanimous inequality}
Paradoxically, power laws, which is the 
current utmost inequality paradigm,
follow from an equanimous consideration
of resources and exhibit other equanimous aspects:
\begin{itemize}
	\item $p(k)=C.k^{-\alpha} \Rightarrow p(k).k^{\alpha}=C$, with C constant. That is, the amount $k$ of resources per component times the amount of those components, which is the total ``instantaneous'' allocated resources, is constant $C$. The scaling factor $\alpha$ is herein interpreted as the number of dimensions in which such resources are being observed.
	\item Each component participates in numerous other complex systems, potentially infinite, and should present a broad, if not complete, sweep of resources allocated to itself.
		These resources are not necessarily of the same type. We assume that human systems, for example, present power-law distributions of knowledge $p_k(k_k)$ and of wealth $p_w(k_w)$, with potentially different (relative) amount $k$ of resources. At the same time, within a fixed type of resource, resources allocated vary in  different systems. For example, an individual tends to have many acquaintances (fixed resource) in its own family, work and neighborhood, a fewer knowns in such circles of distant family members, partners and friends.
	\item The distinction of each component particularities is often not of core importance to describe complex behavior.
		This reflects in symmetries among components. For example, Human individuals form complex social systems with power law distributions of relations. All the participants, by being humans, have the same amount of time available each day, resource, to engage in all the complex systems that are presented by the environment. 
\end{itemize}


\subsection{Wealth distribution}
One manifestation of the power law
which is most fundamental to daily experience
in current society is the inequality of wealth distribution.
There are continuous efforts to deal with this issue,
usually advocating ways to minimize ``social inequality''.
Considering the framework presented withing this letter:

\begin{itemize}
	\item Such an inequality is a natural tendency that follows from presuppositions~\ref{} and~\ref{}, phenomenological mathematical backbone (Section~\ref{}), more than fifty years and wide empirical evidence.

	\item This should make work to diminish it

	\item not any unequal outline, but a power law.
\end{itemize}

In particular, the (publicized)
homogeneity of earnings in public institutions,
and the (publicized) distribution of wealth in whole countries,
reveal that there is indeed efforts to minimize
the strong inequality imposed by power laws.
Additionally, publicized data should be regarded with
extra care and scepticism, as they do not present the
expected power-law distributions.

In particular, power law like inequality seems inevitable,
and a consequence of a distribution of wealth equanimous and insensitive
along wealth allocated to each component.
This implies the necessity of ``work'' for equalization.
Also, we observe that the higher the $k_L$ of equation~\ref{}, the
higher all the probabilistic mass will be located, which
implies greater wealth of the wealthier, ``elites'', hubs.
In other words, the richer the least rich,
the richer the more rich.

\subsection{Naturalization of inequality}

Power laws are very frequent in empirical data.
This already grants its place among the study of natural phenomena.
Many different explanations are given for the many cases where
they are found, with most common denominators being fractals, chaos,
networks. Cases in more traditional fields, such as Newtonian mechanics gravitational force relation to distance with $\alpha=2$ (if masses are fixed), are usually not mentioned in specialized literature. In other words: there seems not to be an unifying theory of why power laws express such
an ubiquitous spectrum of relationships.

If the framework in Section~\ref{sec:form} is valid for all cases where power laws are found,
the consideration of power laws as tied to natural phenomena \emph{per se}, goes a step further. Phenomenologically, yes. That is: if there is
a power law, the analysis developed in Section~\ref{sec:phen} holds.
The power law relation can be regarded as an equanimous distribution
of resources in $\alpha$ dimensions.
If the acting of the ``laws'' given by presuppositions~\ref{} and~\ref{} is fundamentally what is taking place, that should depend on phenomena
and standpoint.
We advocate that this framework deepens the understanding of all power law
incidences and is usually consonant with more explicit and intuitive 
relations of the system, its components and the context.
The core meaning seems to emanate from
the object sizes in the isotropic space.
For the analysis, a reasonable geometric abstraction as such
eases one to grasp the distribution
of fundamental resources of a system of interest.
To relate power laws to the environment is the most effective
way we found to make explicit both the axiomatic
and the phenomenological backbones of power law ubiquity.

% proposta de recursos = E ~ k . N para a lei de potência em sistemas reais

\section{Conclusions}
\subsection*{Acknowledgments}
Authors thank
the General Secretariat of the Republic Presidency (SG-PR) and UNDP for supporting this
research (contract 2013/00056, project BRA/12/018); the National Counsel of Technological 
and Scientific Development (process 140860/2013-4, project 870336/1997-5,
advisor: 
all
labMacambira.sf.net members for pursuing this and other developments;
%----------------------------------------------------------------------------------------
%   BIBLIOGRAPHY
%----------------------------------------------------------------------------------------

%\bibliographystyle{unsrt}
%\bibliographystyle{plain}
\bibliographystyle{ieeetr}
\bibliography{essay}

%----------------------------------------------------------------------------------------

\end{document}
